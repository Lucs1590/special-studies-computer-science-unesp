\newpage
\clearpage
\section{Considerações Finais}
\label{final:final}

Dentre as segmentações disponíveis e estudadas no decorrer desses estudos, entre as segmentações tradicionais (Capítulo \ref{segment:image}), segmentações semânticas (Capítulo \ref{semantic:semantic}), segmentações de instancias (Capítulo \ref{instance:instance}) e segmentações panópticas (Capítulo \ref{panoptic:panoptic}), ficou evidente que a mais próxima do olhar e compreensão humana, que é atingido desde a infância \cite{Mohan2020}, é a segmentação panóptica, visto que classifica todos os pixels de uma cena, além de proporcionar a separação de instâncias, podendo ter uma distinção dentre os objetos de mesma classe.

Dessa forma, as segmentações panópticas são totalmente adequadas para contextos que tenham uma problemática em que todos os pixels de determinada cena possuem relevância e são determinados como região de interesse, o que se aplica para cenas que envolvam imagens médicas relacionadas a tecidos e orgãos em conjunto ou a sistemas e cidades inteligentes, sendo exemplificado por carros autônomos, que necessariamente precisam "estar atentos" a todo novo objeto ou paisagem que se faz presente na cena \cite{Pan2020}.

Todavia, embora as segmentações panópticas sejam as mais adequadas para situações em que todos os pixels necessitam de uma segmentação, vale dizer que a mesma ainda possui pontos a serem relevados. Primeiramente, cita-se o fato das segmentações panópticas possuírem uma abstração até determinado nível quando se trata da segmentação de um objeto da clase \textit{thing}, ou seja, em um caso que há a necessidade de segmentar partes do corpo de uma pessoa, atualmente as segmentações panópticas não podem assim fazer, segmentando a pessoa como um todo. Depois, outro ponto a se comentar está ligado a pequena variação de conjuntos de dados que estejam preparados e anotados para o uso em segmentações panópticas, sendo que os disponíveis estão normalmente relacionados a cidades e instancias pertencentes à mesmas classes.

Assim, levando em conta as considerações do parágrafo anterior, bem como as métricas de qualidades baixas alcançadas pelos modelos de segmentação panóptica observados nas Tabelas \ref{conclusion:table:1}, \ref{conclusion:table:2} e \ref{conclusion:table:1}, o estudo de modificações e unificações nos modelos de segmentação, bem como a aplicação de técnicas artesanais e estudo de conjuntos de dados relacionados a ambientes diferentes do habitual possibilitam a melhoria de técnicas de segmentações panópticas, assim como melhorias relacionadas a indicadores de $PQ$, que podem ser estudados futuramente.

