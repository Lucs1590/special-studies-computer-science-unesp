\section*{Resumo}

Em meio ao campo da visão computacional, atividades relacionadas à segmentação de imagens têm proporcionado avanços quanto a análises médicas mais acuradas, entendimento de cenas, projetos de sistemas autônomos, entre outros estudos semelhantes, os quais têm ganho amplitude devido ao advento das redes neurais artificiais e das técnicas de aprendizado profundo, que proporcionam base para o desenvolvimento de muitos modelos e arquiteturas que almejam alcançar o estado-da-arte, proporcionando melhores desempenhos para as questões de segmentação de imagens.
Destarte, este trabalho tem por objetivo introduzir o contexto e conceitos de redes neurais artificiais e redes neurais convolucionais, apresentando as suas vantagens e desvantagens mediante a algumas problemáticas, além de apresentar as técnicas de segmentação tradicionais, que servem de base para a formação de novas técnicas, bem como apresentar técnicas de segmentação mais modernas, que trabalham com aprendizado profundo e promovem uma aproximação no que se refere ao olhar e entendimento humano, proporcionando uma maior riqueza de informações e semântica às cenas analisadas. 

