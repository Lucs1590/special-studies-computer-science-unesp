\section*{Resumo}
\makeatletter
\newcommand*{\rom}[1]{\expandafter\@slowromancap\romannumeral #1@}
\makeatother
Em meio ao campo da visão computacional, atividades relacionadas à segmentação de imagens têm proporcionado avanços quanto a análises médicas mais acuradas, entendimento de cenas, projetos de sistemas autônomos, entre outros estudos semelhantes, os quais têm ganho amplitude devido ao advento das redes neurais artificiais e das técnicas de aprendizado profundo, que proporcionam base para o desenvolvimento de muitos modelos e arquiteturas que almejam alcançar o estado-da-arte, proporcionando melhores desempenhos para as questões de segmentação de imagens.
Dentre as aplicações na área de segmentação, destaca-se o campo odontológico, em que estudos e ferramentas - desde os tradicionais aos mais modernos - vêm sendo desenvolvidos com a meta de auxiliar os profissionais da área, proporcionando agilidade, detalhamento e assertividade quanto à situação odontológica de cada paciente, como quantidade de dentes presentes, dentes ausentes, presença de cáries, área de comprometimento dos dentes e componentes visuais presentes na boca. 
Dentre os trabalhos de segmentação com a aplicação na área odontológica produzidos, é notoriamente baixa a quantidade de trabalhos que explora as segmentações mais modernas, as quais são capazes de proporcionar uma aproximação no que se refere ao entendimento humano e realmente auxiliar em detalhes que podem passar desapercebidos, assim, de fato, proporcionando uma maior riqueza de informações e semântica às cenas analisadas.
Destarte, o presente trabalho propõe apresentar uma visão geral sobre conceitos de redes neurais artificiais e contextualizar técnicas de segmentações tradicionais e modernas, apresentando suas vantagens e desvantagens mediante a algumas problemáticas, objetivando alvitrar: \rom{1}) a aplicação de uma segmentação moderna em componentes visuais odontológicos descrevendo uma cena hierarquicamente e \rom{2}) alterar a camada de \textit{pooling} dos modelos base de segmentação moderna, de efeito a contribuir com profissionais odontológicos em sua análise e aumentar as métricas de avaliação dos modelos, respectivamente.
Por fim, espera-se que esta proposta possa contribuir para a criação de uma ferramenta de análise automática de saúde bucal, que possa ser utilizada por profissionais ou indivíduos, a fim de se ter uma ideia da condição bucal do paciente, permitindo assim acesso à um análise odontológica inicial e monitoramento da saúde bucal da população.
